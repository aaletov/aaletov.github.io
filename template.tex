\DocumentMetadata{pdfstandard=A-2b}
\documentclass[letterpaper,11pt]{article}

\usepackage{luacode}
\usepackage{latexsym}
\usepackage[empty]{fullpage}
\usepackage{titlesec}
\usepackage{marvosym}
\usepackage[usenames,dvipsnames]{color}
\usepackage{verbatim}
\usepackage{enumitem}
\usepackage[hidelinks]{hyperref}
\usepackage{fancyhdr}
\usepackage{fontspec}
\setmainfont{Noto Serif}
\usepackage[english, russian]{babel}
\usepackage{tabularx}
\usepackage{graphicx}
\usepackage{changepage}
\usepackage{blindtext}


\pagestyle{fancy}
\fancyhf{} % clear all header and footer fields
\fancyfoot{}
\renewcommand{\headrulewidth}{0pt}
\renewcommand{\footrulewidth}{0pt}

% Adjust margins
\addtolength{\oddsidemargin}{-0.5in}
\addtolength{\evensidemargin}{-0.5in}
\addtolength{\textwidth}{1in}
\addtolength{\topmargin}{-.5in}
\addtolength{\textheight}{1.0in}

\urlstyle{same}

\raggedbottom
\raggedright
\setlength{\tabcolsep}{0in}

% Sections formatting
\titleformat{\section}{
  \vspace{-4pt}\scshape\raggedright\large
}{}{0em}{}[\color{black}\titlerule \vspace{-5pt}]


%-------------------------
% Custom commands
\newcommand{\resumeItem}[1]{
  \item\small{
    {#1 \vspace{-2pt}}
  }
}

\newcommand{\resumeSubheading}[4]{
  \vspace{-2pt}\item
    \begin{tabular*}{0.97\textwidth}[t]{l@{\extracolsep{\fill}}r}
      \textbf{#1} & #2 \\
      #3 & #4 \\
    \end{tabular*}\vspace{-7pt}
}

\newcommand{\resumeSubSubheading}[2]{
    \item
    \begin{tabular*}{0.97\textwidth}{l@{\extracolsep{\fill}}r}
      \textit{\small#1} & \textit{\small #2} \\
    \end{tabular*}\vspace{-7pt}
}

\newcommand{\resumeProjectHeading}[2]{
    \item
    \begin{tabular*}{0.97\textwidth}{l@{\extracolsep{\fill}}r}
      \small#1 & #2 \\
    \end{tabular*}\vspace{-7pt}
}

\newcommand{\resumeSubItem}[1]{\resumeItem{#1}\vspace{-4pt}}

\renewcommand\labelitemii{$\vcenter{\hbox{\tiny$\bullet$}}$}

\newcommand{\resumeSubHeadingListStart}{\begin{itemize}[leftmargin=0.15in, label={}]}
\newcommand{\resumeSubHeadingListEnd}{\end{itemize}}
\newcommand{\resumeItemListStart}{\begin{itemize}}
\newcommand{\resumeItemListEnd}{\end{itemize}\vspace{-5pt}}

%-------------------------------------------
%%%%%%  RESUME STARTS HERE  %%%%%%%%%%%%%%%%%%%%%%%%%%%%

\begin{document}

\begin{luacode}
json = require "json"

function texio.debug_nl (str)
    texio.write_nl("-----> " .. str)
    texio.write_nl("^^^")
end

function readJSON (filename)
    local f = io.open(filename, "r")
    local s = f:read("*all")
    f:close()
    return s
end

function tex.printf (pattern, str)
    local formatted = string.format(pattern, str)
    tex.print(formatted)
end

local jsonString = readJSON("resume/ru/resume.json")
decodedJSON = json.decode(jsonString)

texio.debug_nl("Object type is: " .. type(decodedJSON))

function get_location ()
    return table.concat({decodedJSON.basics.location.city, decodedJSON.basics.location.region}, ", ")
end

function get_mail ()
    local mail = decodedJSON.basics.email
    local href = string.format("\\href{mailto:%s}{\\underline{%s}}", mail, mail)
    return href
end

function make_href (link)
    print_link = string.gsub(link, "https://", "")
    print_link = string.gsub(print_link, "www.", "")
    local href = string.format("\\href{https://%s}{\\underline{%s}}", link, print_link)
    return href
end

function get_profiles ()
    local contacts = ""
    -- while line not exceeded
    for i, profile in ipairs(decodedJSON.basics.profiles) do
        contacts = contacts .. make_href(profile.url)
        if i ~= table.getn(decodedJSON.basics.profiles) then
            contacts = contacts .. " $|$ "
        end
    end
    return contacts
end

function parse_date (date)
    return {
        year=string.sub(date, 1, 4),
        month=string.sub(date, 6, 7),
        day=string.sub(date, 9, 10)
    }
end

function to_month_date (date)
    local month_from_number = {
        ["01"]="Январь",
        ["02"]="Февраль",
        ["03"]="Март",
        ["04"]="Апрель",
        ["05"]="Май",
        ["06"]="Июнь",
        ["07"]="Июль",
        ["08"]="Август",
        ["09"]="Сентябрь",
        ["10"]="Октябрь",
        ["11"]="Ноябрь",
        ["12"]="Декабрь"
    }
    return string.format("%s %s", month_from_number[date.month], date.year)
end


function make_education_heading (education)
    local institution = education.institution
    local location = ""
    local faculty = table.concat({education.studyType, education.area}, ", ")
    local start_year = parse_date(education.startDate).year
    local end_year = parse_date(education.endDate).year
    local dates = table.concat({start_year, end_year}, "-")
    local str_heading = string.format("\\resumeSubheading{%s}{%s}{%s}{%s}", institution, location, faculty, dates)
    return str_heading
end

function make_education_courses (education)
    local heading = "\\resumeProjectHeading{\\textbf{Курсы:}}{ }"
    local item_list_start = "\\resumeItemListStart"
    local item_list_end = "\\resumeItemListEnd"
    local courses = ""
    for i, course in ipairs(education.courses) do
        courses = courses .. string.format("\\resumeItem{%s}", course)
    end
    local str_courses = table.concat({heading, item_list_start, courses, item_list_end}, " ")
    return str_courses
end

function get_education ()
    local education = table.concat({make_education_heading(decodedJSON.education[1]), make_education_courses(decodedJSON.education[1])}, "\\\\")
    return education
end

function make_project_heading (project)
    local name = string.format("\\href{%s}{%s}", project.url, project.name)
    local parsed_start_date = (project.startDate) and to_month_date(parse_date(project.startDate)) or ""
    local parsed_end_date = (project.endDate) and to_month_date(parse_date(project.endDate)) or ""
    local dates = table.concat({parsed_start_date, parsed_end_date}, " -- ")
    local str_heading = string.format("\\resumeProjectHeading{\\textbf{%s}}{%s}", name, dates)
    return str_heading
end

function make_project_highlights (project)
    if project.highlights == nil then
        return ""
    end
    local item_list_start = "\\resumeItemListStart"
    local item_list_end = "\\resumeItemListEnd"
    local highlights = string.format("\\resumeItem{%s}", project.description)
    for i, highlight in ipairs(project.highlights) do
        highlights = highlights .. string.format("\\resumeItem{%s}", highlight)
    end
    local str_highlights = table.concat({item_list_start, highlights, item_list_end}, " ")
    return str_highlights
end

function make_project (project)
    return string.format("%s %s", make_project_heading(project), make_project_highlights(project))
end

function get_projects ()
    local projects = ""
    for i, project in ipairs(decodedJSON.projects) do
        projects = projects .. make_project(project)
    end
    return projects
end

function make_job_heading (job)
    local name = string.format("\\href{%s}{%s}", job.url, job.name)
    local location = job.location
    local position = job.position
    local parsed_start_date = (job.startDate) and to_month_date(parse_date(job.startDate)) or ""
    local parsed_end_date = (job.endDate) and to_month_date(parse_date(job.endDate)) or ""
    local dates = table.concat({parsed_start_date, parsed_end_date}, " -- ")
    local str_heading = string.format("\\resumeSubheading{%s}{%s}{%s}{%s}", name, location, position, dates)
    return str_heading
end

function make_job (job)
    return string.format("%s %s", make_job_heading(job), make_project_highlights(job))
end

function get_jobs ()
    local jobs = ""
    for i, job in ipairs(decodedJSON.work) do
        jobs = jobs .. make_job(job)
    end
    return jobs
end

function get_skills ()
    local skills = ""
    for i, skill in ipairs(decodedJSON.skills) do
        local area = skill.name
        local subskills = table.concat(skill.keywords, ", ")
        skills = skills .. string.format("\\textbf{%s}{: %s}", area, subskills)
        if i ~= table.getn(decodedJSON.skills) then
            skills = skills .. "\\\\"
        end
    end
    return skills
end
\end{luacode}


\begin{center}
    \textbf{\Huge \scshape \directlua{ tex.print(decodedJSON.basics.name) }} \\ \vspace{1pt}
    % Research ways to use inline luacode properly
    $|$ \small \directlua{ tex.print(get_location()) } $|$
    \small \directlua{ tex.print(decodedJSON.basics.phone) } $|$  \directlua{ tex.print(get_mail()) } $|$
    \directlua{ tex.print(get_profiles()) }
\end{center}

%-----------ABOUT ME------------
\section{О себе}
\begin{itemize}[leftmargin=0.15in, label={}]
    \item \directlua{ tex.print(decodedJSON.basics.summary) }
\end{itemize}

%-----------EDUCATION-----------
\section{Образование}
  \resumeSubHeadingListStart
    \directlua{ tex.print(get_education()) }

  \resumeSubHeadingListEnd

%-----------EXPERIENCE-----------
\section{Опыт}
    \resumeSubHeadingListStart
        \directlua{ tex.print(get_jobs()) }
    \resumeSubHeadingListEnd

%-----------PROJECTS-----------
\section{Проекты}
    \resumeSubHeadingListStart
        \directlua{ tex.print(get_projects()) }
    \resumeSubHeadingListEnd

%-----------PROGRAMMING SKILLS-----------
\section{Технические навыки}
 \begin{itemize}[leftmargin=0.15in, label={}]
    \small{\item{
        \directlua{ tex.print(get_skills()) }
    }}
 \end{itemize}

\end{document}
